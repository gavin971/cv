\documentclass[letterpaper,11pt]{article}
\usepackage{hyperref}
\usepackage{geometry}

\usepackage{sectsty}
\sectionfont{\rmfamily\mdseries\scshape\Large}
\subsectionfont{\rmfamily\mdseries\itshape\large}

\def\name{Matthew R. Terry, Ph.D.}

\setlength{\parskip}{\baselineskip}
\setlength{\parindent}{0pt}

\begin{document}

\name

\vspace{-0.17in}
\begin{tabular*}{\textwidth}{@{}l @{\extracolsep{\fill}} r}
	1612 Parker St. Apt D  	& 608.658.4316 \\
	Berkeley, CA 94703 		& \href{mailto:matt.terry@gmail.com}{matt.terry@gmail.com}
\end{tabular*}

\hrule

\vspace{0.8in}

To whom it may concern:

I am interested in applying for your open developer position.

While my background is in a somewhat different field (plasma physics) we share
interest high level tools to solve interesting, data-driven problems.
I am a seasoned scientific Python developer with extensive
experience in making high level computing tools go fast;  I'm a big fan of
tools like numpy, scipy, cython, and such to build high level interfaces to
high performance compiled kernels; and I'm comfortable working with large
datasets and parallel environments.  Recently, my work has involved methods for
applying numerical optimization tools to automate parts of the scientific
workflow, as well as designing and analyzing physics experiments.

I value scientific reproducibility, open software infrastructure, and
community.  This is reflected in my roles as an organizer of the SciPy
conference and member of the NumFOCUS Foundation.  I think these values to be
good fits for your analytic efforts as well as its use of the open
Python computing ecosystem.


Sincerely,

Matt Terry

\vspace{0.5in}
\end{document}

