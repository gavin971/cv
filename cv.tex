% LaTeX Curriculum Vitae Template
%
% Copyright (C) 2004-2009 Jason Blevins <jrblevin@sdf.lonestar.org>
% http://jblevins.org/projects/cv-template/
%
% You may use use this document as a template to create your own CV
% and you may redistribute the source code freely. No attribution is
% required in any resulting documents. I do ask that you please leave
% this notice and the above URL in the source code if you choose to
% redistribute this file.

\documentclass[letterpaper,11pt]{article}

\usepackage{hyperref}
\usepackage[margin=1.2in]{geometry}
\usepackage{amsmath}

% Comment the following lines to use the default Computer Modern font
% instead of the Palatino font provided by the mathpazo package.
% Remove the 'osf' bit if you don't like the old style figures.
%\usepackage[T1]{fontenc}
%\usepackage[sc,osf]{mathpazo}

\usepackage[resetlabels]{multibib}

% Set your name here
\def\name{Matthew R. Terry, Ph.D.}

% The following metadata will show up in the PDF properties
\hypersetup{%
  colorlinks = false,
  urlcolor = black,
  pdfauthor = {\name},
  pdfkeywords = {plasma physics, computational physics, python},
  pdftitle = {\name: Curriculum Vitae},
  pdfsubject = {Curriculum Vitae},
  pdfpagemode = UseNone
}

% Customize page headers
\pagestyle{myheadings}
\markright{\name}
\thispagestyle{empty}

% Custom section fonts
\usepackage{sectsty}
\sectionfont{\rmfamily\mdseries\scshape\Large}
\subsectionfont{\rmfamily\mdseries\itshape\large}

% Other possible font commands include:
% \ttfamily for teletype,
% \sffamily for sans serif,
% \bfseries for bold,
% \scshape for small caps,
% \normalsize, \large, \Large, \LARGE sizes.

% Don't indent paragraphs.
\setlength\parindent{0em}

\newcites{%
	ref,%
	invtalk%
	}{%
	Refereed Journals,%
	Invited Talks%
}

\newcommand{\sectionline}{\vspace{-0.05in}\hrule\vspace{0.05in}}
\setlength{\parskip}{0.3em}

\usepackage[compact]{titlesec}
\titlespacing{\section}{0em}{1em}{0.5em}


\begin{document}

\begin{center}
	{\huge\scshape \name}

	1612 Parker St. Apt D $\bullet$
	Berkeley, CA 94703 $\bullet$
	608.658.4316 $\bullet$
	\href{mailto:me@mattterry.net}{me@mattterry.net}
\end{center}

\section*{Objective}
\sectionline
I am seeking a position employing computational tools in the analysis and
solution of scientific and technical problems.  I am a US citizen and currently
hold a Department of Energy ``Q'' clearance.

\section*{Education and Training}
\sectionline
\begin{itemize}
	\item
		\textbf{Lawrence Livermore National Laboratory} Post-docoral research, 2010-present \\
		Inertial Fusion and High Energy Density Physics Target Design
	\item 
		\textbf{University of Wisconsin-Madison} Ph.D. Nuclear Engineering, 2010 \\
		``Effect of Different Charged Particle Stopping Power Models on ICF Ignition''
	\item 
		\textbf{University of Wisconsin-Madison} M.S. Nuclear Engineering, 2006
	\item 
		\textbf{Georgia Institute of Technology} B.S. Nuclear and Radiological Engineering, 2004
\end{itemize}



\section*{Computational Physicist Engineer}
\sectionline
I am by temperament an engineer, by training a physicist, and by practice a
software developer.  For the last 9 years, my research has focused on using
practical
computational methods to model and understand the behavior of high energy
density plasmas, that is, ionized matter with pressure greater than $>10^6$ time atmospheric
pressure and temperatures exceeding $10^4$K.  I have studied in detail, shock hydrodynamics,
fusion burn kinetics, and inertial fusion design.

Intrinsic to my scientific research is the development of software.  Among my
software projects I have:
\begin{enumerate}
	\item Developed Monte Carlo charged particle transport packages for two
		radiation hydrodynamics codes ``Bucky'' and ``Draco''.  These were
		developed to study the physics of fusion burn kinetics in high density
		plasmas.
	\item Developed a platform to automate the design of certain aspect of an
		inertial fusion target design.  It uses parallel optimization
		techniques to optimize laser timing and intensity features.  In
		practice, multiple-day exercise are replaced by a single 4 hour batch
		job.
\end{enumerate}

I am conversant in numerical methods for solving partial differential.
I have significant experience with Lagrangian hydrodynamics, Monte Carlo
solutions to the Boltzmann Equation, and compatible finite different
methods.  I am interested in large scale numerical
optimization, especially techniques using concurrent function evaluations.


\section*{High Performance Python Expert}
\sectionline
I have been developing scientific software for more than a decade and Python
for nearly 9 years.  I have experience running Python in parallel environments
on large computing clusters and am comfortable processing ``big data''.  My
scientific analysis tools are all developed in Python and make extensive use of
the NumPy/SciPy ecosystem as well as interfacing with external code/libraries
when appropriate.  I have extensive experience developing convenient Python
interfaces to high performance, compiled libraries using tools such as Cython,
f2py, and SWIG.

Among my Python projects I have:
\begin{enumerate}
	\item Developed an analysis library for pre-processing, post-processing,
		and embedded steering of the radiation-hydrodynamics program HYDRA.
	\item Currently developing a program ``Yoink'' to reverse-engineering data
		from scientific figures.
	\item Converted a rigid legacy Fortran program to a script-able
		Python-based program without significant loss of performance.
\end{enumerate}

In addition to Python, I have worked with C, C++, Fortran (legacy Fortran 77
and modern Fortran 90+), and Go.  I am a vim, git, and \LaTeX user and am at
home with the command-line.


\section*{Scientific Computing Community Advocate}
\sectionline
I am an advocate for computing competency in the sciences.  While in graduate
school, I helped found The Hacker Within, a peer learning group for scientific
computing and open source tools.  Additionally, I teach scientists software
development skills as an instructor with  Software Carpentry.

I am an active member of the Python scientific computing community.   As a
member of the program committee, I helped set the program for the Scientific
Computing in Python conference.  I am a member of the UC-Berkely py4science
community and a member of NumFocus, a Python oriented scientific
computing advocacy organization.


%\section*{Other}
%\sectionline
%Self motivated and a problem solver.  LDRD\@.  I work best within a community 

\pagebreak

\section*{Publications}
\hrule
\vspace{0.05in}

% papers
\nociteref{Terry2012b}
\nociteref{Koning2011}
\nociteref{Barnard2013}
\nociteref{Terry2010}
\nociteref{Stacey2005}

% Invited Talks
\nociteinvtalk{Terry2012a}
\nociteinvtalk{Terry2013_usf}


\bibliographystyleref{plainyr-rev}
\bibliographyref{/Users/terry10/doc/latex_files/library}

\bibliographystyleinvtalk{plainyr-rev}
\bibliographyinvtalk{/Users/terry10/doc/latex_files/library}

\end{document}
