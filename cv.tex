% LaTeX Curriculum Vitae Template
%
% Copyright (C) 2004-2009 Jason Blevins <jrblevin@sdf.lonestar.org>
% http://jblevins.org/projects/cv-template/
%
% You may use use this document as a template to create your own CV
% and you may redistribute the source code freely. No attribution is
% required in any resulting documents. I do ask that you please leave
% this notice and the above URL in the source code if you choose to
% redistribute this file.

\documentclass[letterpaper,11pt]{article}

\usepackage{hyperref}
\usepackage{geometry}

% Comment the following lines to use the default Computer Modern font
% instead of the Palatino font provided by the mathpazo package.
% Remove the 'osf' bit if you don't like the old style figures.
%\usepackage[T1]{fontenc}
%\usepackage[sc,osf]{mathpazo}

\usepackage[resetlabels]{multibib}

% Set your name here
\def\name{Matthew R. Terry}

% The following metadata will show up in the PDF properties
\hypersetup{%
  colorlinks = false,
  urlcolor = black,
  pdfauthor = {\name},
  pdfkeywords = {plasma physics, computational physics, python},
  pdftitle = {\name: Curriculum Vitae},
  pdfsubject = {Curriculum Vitae},
  pdfpagemode = UseNone
}

% Customize page headers
\pagestyle{myheadings}
\markright{\name}
\thispagestyle{empty}

% Custom section fonts
\usepackage{sectsty}
\sectionfont{\rmfamily\mdseries\scshape\Large}
\subsectionfont{\rmfamily\mdseries\itshape\large}

% Other possible font commands include:
% \ttfamily for teletype,
% \sffamily for sans serif,
% \bfseries for bold,
% \scshape for small caps,
% \normalsize, \large, \Large, \LARGE sizes.

% Don't indent paragraphs.
\setlength\parindent{0em}

\newcites{%
	ref,%
	refproc,%
	invtalk,%
	proc,%
	thesis%
	}{%
	Refereed Journals,%
	Refereed Conference Proceedings,%
	Invited Talks,%
	Conference Proceedings,%
	Ph.D. Thesis%
}

\begin{document}

\begin{center}
	{\huge\scshape \name}

	\textit{%
		2223 Bonar St. Apt E $\bullet$
		Berkeley, CA 94702 $\bullet$
		608.658.4316 $\bullet$
		\href{mailto:matt.terry@gmail.com}{matt.terry@gmail.com}
	}
\end{center}

%\section*{Objective}
%\hrule
%\vspace{0.05in}
%I am seeking a position involving the application of computational tools in the analysis and solution of scientific or technical problems.

\section*{Personal Information}
\hrule
\vspace{0.05in}
\begin{itemize}
	\item United States Citizen
	\item US Department of Energy ``Q'' Clearance
\end{itemize}


\section*{Education}
\hrule
\vspace{0.05in}
\begin{itemize}
	\item 
		\textbf{University of Wisconsin-Madison} \\
		Ph.D. Nuclear Engineering \\
  		Thesis: Effect of Different Charged Particle Stopping Power Models on ICF Ignition \\
		Advisor: Gregory A. Moses
	\item 
		\textbf{University of Wisconsin-Madison} \\
		M.S. Nuclear Engineering \\
		Graduated May 2006, 3.53 GPA 
	\item 
		\textbf{Georgia Institute of Technology} \\
		B.S. Nuclear and Radiological Engineering with Honors \\
		Graduated May 2004, 3.77 GPA
\end{itemize}


%\section*{Employment}
%\begin{itemize}
%\end{itemize}


\section*{Publications}
\hrule
\vspace{0.05in}

% Referreed papers
\nociteref{Terry2012b}
\nociteref{Stacey2005}

% Referreed proceedings
\nociterefproc{Koning2011}

% Invited Talks
\nociteinvtalk{Terry2012a}

% Thesis
\nocitethesis{Terry2010}

% Conference Proceedings
\nociteproc{Terry2012}
\nociteproc{Terry2011}
\nociteproc{Terry2010a}
% OLUG
\nociteproc{Terry2009}
\nociteproc{Terry2004}


\bibliographystyleref{plainyr-rev}
\bibliographyref{/Users/terry10/doc/latex_files/library}

\bibliographystylerefproc{plainyr-rev}
\bibliographyrefproc{/Users/terry10/doc/latex_files/library}

\bibliographystyleinvtalk{plainyr-rev}
\bibliographyinvtalk{/Users/terry10/doc/latex_files/library}

\bibliographystylethesis{plainyr-rev}
\bibliographythesis{/Users/terry10/doc/latex_files/library}

\bibliographystyleproc{plainyr-rev}
\bibliographyproc{/Users/terry10/doc/latex_files/library}


\section*{Technical Experience}
\hrule
\vspace{0.05in}

\subsection*{Post-Doctoral Research}
My post-doctoral research has focused on the design of novel inertial
confinement fusion targets and on designing high energy density physics
experiments.  I developed a shock ignition target compatible with ``day one''
NIF hardware as well as developed a direct drive heavy ion driven ignition
target that makes use of a long range heavy ion deposition and tamped ablation
to achive high drive efficiency.  Additional work has expended into the design
of laser driven hydrodynamics experiments intended to test the physics of the
ion driven X-target, the modeling of a high energy backlighter for NIF, and
experiments measuring the dynamic response of BCC metals to shocks.


\subsection*{Ph.D. Research}
My Ph.D. research studied the effect of different charged particle stopping
power models on the performance of inertial confinement fusion (ICF) targets.
Much work went into the development of a library (``Deeks'') that implements
many different models and does extensive consistency checking between the model
and the conditions to which it is applied.  This library has been integrated
into the pre-existing time dependent charged particle tracking (a Monte
Carlo-like transport scheme) package in an existing multi-physics code
(``Bucky'').  The integration allows me to examine the effect of different
transport models in integrated realistic simulations.

\vspace{0.1in}

More specifically, my research involves studying the effect collision operators have on transport properties of the plasma.  Due to the particularly high densities of ICF plasmas, this requires a careful, combined treatment of both the collisional and dielectric properties of the plasma, that is use of so-called ``convergent'' kinetic theories.  I have implemented stopping power models by 
Landau; 
Spitzer; 
Jackson; 
Brysk; 
Skupsky; 
Kihara and Aono; 
May and Cramer; 
Li and Petrasso; 
and 
Brown, Preston, and Singleton%
.
In my own research, I have derived and implemented Fermi-degenerate extensions of the Landau and Li and Petrasso models.  I am currently composing papers for publication on my Fermi-degenerate extension work and another on my analysis of the applicability of existing stopping power models in ICF ignition experiments.  


\subsection*{Software Development}
Over the course of my research, I have developed a significant amount of
scientific software.  I am a skilled Python developer, with 8 years of
experience with NumPy, SciPy, Matplotlib, Cython, PyTables, etc.  I am
proficient in Fortran, C, and C++ and have experience working in parallel
environments.  I am a vim, git, and \LaTeX user and am at home with the
command-line.

The computing aspect of my Ph.D. research required the integration of a library
written in C++ (Deeks) with an existing Fortran program (Bucky).  Data analysis
in compiled languages is overly time intensive, so I developed Python
interfaces to both codes.  The flexibility provided by the Python interface
enabled the rapid development of an automatic shock tuner for Bucky.  The tuner
completely automates a previously labor and analysis intensive process.  What
once took a week, can now be accomplished in hours.

\subsection*{Undergraduate Research at LLNL}
\begin{itemize}
	\item Summers 2003, 2004 - Worked with Ogden Jones on the analysis of ICF thinshell symmetry experiments using view-factor codes and radiation hydrodynamics code Hydra.
	\item Summer 2002 -  Worked with Dmitri Ryutov on magnetic islands in SSPX
\end{itemize}

\subsection*{Professional Societies}
\begin{itemize}
	\item The American Physical Society
\end{itemize}

\pagebreak

\section*{Technical Leadership}
\hrule
\vspace{0.05in}

\subsection*{Scientific Programming in Python Program Committee}
The Program Commitee for the Scientific Programming in Python Conference is
responsible for reviewing and selecting talks from submitted proposals.
Committee is also responsible for reviewing submitted papers submitted to the
conference proceedings.

\subsection*{The Hacker Within}
The Hacker Within THW is a student-led, skill sharing interest group for
scientific software development.  It was founded in 2008 by two colleagues and
myself.  We hold bi-weekly meetings and hold occasional ``boot camps,'' highly
focused multi-day workshops.  Our most recent boot camp on the Python
programming language (\url{http://python.hackerwithin.org}) had more than 70
attendees from 20 departments and included more than 18 hours of instruction.
In addition to organization, I was responsible for teaching the sessions
``Modules, Scripts, Packages and Classes,''  ``Graphing with MatPlotLib,''
``Extending python with C/C++ using SWIG,''  and ``Batteries Included - The
Python Standard Library.''  The Hacker Within collaborates with
\href{software-carpentry.org}{Software Carpentry} to teach THW-style workshops
at universities and laboratories across the US and Europe.

\section*{Community Leadership}
\hrule
\vspace{0.05in}
\subsection*{The Journey Community Church Council}
From 2009-2010, I was been one of 5 members of The Journey Community's Church
Council.  Members of the council, are responsible for overseeing the church's
finances.  They are responsible for setting the  budget for the church and
making financial decisions for the church.  Additionally, they oversee the
finances of \href{http://www.beautifulchild.org}{Beautiful Child}, a charity
that sponsors an orphanage in Jacmel, Haiti.

\subsection*{Jumptown}
From 2002-2004 I was president of a swing dancing club.  I was responsible for
organizing weekly classes, monthly dances and semesterly workshops.  Under my
supervision the club grew from about 20 people to more than 100 and oversaw a
budget of several thousand dollars.

\section*{Awards}
\hrule
\vspace{0.05in}
\begin{itemize}
	\item National Physical Sciences Consortium Fellowship in conjunction with Sandia National Laboratories 2007-2010
	\item James Poukey Fellowship 2006
	\item National Academy for Nuclear Training Scholarship 2002-2004
	\item American Nuclear Society Scholarship 2002-2003
\end{itemize}

\bigskip

\end{document}
