% LaTeX Curriculum Vitae Template
%
% Copyright (C) 2004-2009 Jason Blevins <jrblevin@sdf.lonestar.org>
% http://jblevins.org/projects/cv-template/
%
% You may use use this document as a template to create your own CV
% and you may redistribute the source code freely. No attribution is
% required in any resulting documents. I do ask that you please leave
% this notice and the above URL in the source code if you choose to
% redistribute this file.

\documentclass[letterpaper,11pt]{article}

\usepackage{hyperref}
\usepackage{geometry}
\usepackage{amsmath}

% Comment the following lines to use the default Computer Modern font
% instead of the Palatino font provided by the mathpazo package.
% Remove the 'osf' bit if you don't like the old style figures.
%\usepackage[T1]{fontenc}
%\usepackage[sc,osf]{mathpazo}

\usepackage[resetlabels]{multibib}

% Set your name here
\def\name{Matthew R. Terry, Ph.D.}

% The following metadata will show up in the PDF properties
\hypersetup{%
  colorlinks = false,
  urlcolor = black,
  pdfauthor = {\name},
  pdfkeywords = {plasma physics, computational physics, python},
  pdftitle = {\name: Curriculum Vitae},
  pdfsubject = {Curriculum Vitae},
  pdfpagemode = UseNone
}

% Customize page headers
\pagestyle{myheadings}
\markright{\name}
\thispagestyle{empty}

% Custom section fonts
\usepackage{sectsty}
\sectionfont{\rmfamily\mdseries\scshape\Large}
\subsectionfont{\rmfamily\mdseries\itshape\large}

% Other possible font commands include:
% \ttfamily for teletype,
% \sffamily for sans serif,
% \bfseries for bold,
% \scshape for small caps,
% \normalsize, \large, \Large, \LARGE sizes.

% Don't indent paragraphs.
\setlength\parindent{0em}

\newcites{%
	ref,%
	refproc,%
	invtalk,%
	proc,%
	thesis%
	}{%
	Refereed Journals,%
	Refereed Conference Proceedings,%
	Invited Talks,%
	Conference Proceedings,%
	Ph.D. Thesis%
}

\begin{document}

\begin{center}
	{\huge\scshape \name}

	\textit{%
		1612 Parker St. Apt D $\bullet$
		Berkeley, CA 94703 $\bullet$
		608.658.4316 $\bullet$
		\href{mailto:me@mattterry.net}{me@mattterry.net}
	}
\end{center}

\section*{Objective}
\hrule
\vspace{0.05in}
I am seeking a position involving the application of computational tools in the analysis and solution of scientific and technical problems.

\section*{Personal Information}
\hrule
\vspace{0.05in}
\begin{itemize}
	\item United States Citizen
	\item US Department of Energy ``Q'' Clearance
\end{itemize}


\section*{Education}
\hrule
\vspace{0.05in}
\begin{itemize}
	\item
		\textbf{Lawrence Livermore National Laboratory} \\
		Post-doctoral research \\
		Fusion Energy Program \\
		Advisors: L. John Perkins, John J. Barnard, Alex Friedman
	\item 
		\textbf{University of Wisconsin-Madison} \\
		Ph.D. Nuclear Engineering \\
  		Thesis: Effect of Different Charged Particle Stopping Power Models on ICF Ignition \\
		Advisor: Gregory A. Moses
	\item 
		\textbf{University of Wisconsin-Madison} \\
		M.S. Nuclear Engineering \\
		Graduated May 2006, 3.53 GPA 
	\item 
		\textbf{Georgia Institute of Technology} \\
		B.S. Nuclear and Radiological Engineering with Honors \\
		Graduated May 2004, 3.77 GPA
\end{itemize}


%\section*{Employment}
%\begin{itemize}
%\end{itemize}


\section*{Publications}
\hrule
\vspace{0.05in}

% Referreed papers
\nociteref{Terry2012b}
\nociteref{Stacey2005}

% Referreed proceedings
\nociterefproc{Koning2011}

% Invited Talks
\nociteinvtalk{Terry2012a}
\nociteinvtalk{Terry2013_usf}

% Thesis
\nocitethesis{Terry2010}

% Conference Proceedings
\nociteproc{Terry2012}
\nociteproc{Terry2011}
\nociteproc{Terry2010a}
% OLUG
\nociteproc{Terry2009}
\nociteproc{Terry2004}


\bibliographystyleref{plainyr-rev}
\bibliographyref{/Users/terry10/doc/latex_files/library}

\bibliographystylerefproc{plainyr-rev}
\bibliographyrefproc{/Users/terry10/doc/latex_files/library}

\bibliographystyleinvtalk{plainyr-rev}
\bibliographyinvtalk{/Users/terry10/doc/latex_files/library}

\bibliographystylethesis{plainyr-rev}
\bibliographythesis{/Users/terry10/doc/latex_files/library}

\bibliographystyleproc{plainyr-rev}
\bibliographyproc{/Users/terry10/doc/latex_files/library}


\section*{Technical Experience}
\hrule
\vspace{0.05in}

\subsection*{Post-Doctoral Research}
My post-doctoral research has focused on understanding the physics of objects
subjected to extremely high power ($10^{14}$ W), high intensity ($10^{15}
\text{W}/\text{cm}^2$) laser and particle beams.  My research in the field of
high energy density physics has focused on two broad topics: design of inertial
confinement fusion (ICF) targets and design and analysis of materials science
experiments involving metals at high pressure ($100 - 5000$ kBar).

My fusion efforts have all occurred in the field of inertial confinement fusion
(ICF), a potential means of energy production that uses micro-explosions to
produce conditions with densities 100,000 times solid density and temperatures
in excess of $10^8$ K.  With L. J. Perkins and S. Sepke of AX Division, I developed a shock
ignition target compatible with ``day one'' hardware on the recently completed
National Ignition Facility (NIF) in Livermore, CA\@.  The design includes an
assessment of energetics, polar drive efficiency, hydrodynamic stability, and
performance degradation due to experimental jitter and hot electrons.  This
research resulted in the Physics of Plasmas paper ``Design of a DT-Ablator
Shock Ignition Target for
the National Ignition Facility.''

Working with the Heavy Ion Fusion Virtual National Laboratory and LLNL's Fusion
Energy Program, I developed heavy ion driven ICF concept. It uses directly
incident ion beams on a spherical target to drive an ICF implosion.  This
design makes use of the deep, spatially peaked deposition of heavy ion beams
combined with the tamping effect of the material upstream.  For the
acceleration
stage, the drive transitions to radiation driven ablation, with the tamper
material now acting as a spherical hohlraum.  A paper describing the design,
including an assessment of the hydrodynamic stability, is in preparation.  

In conjunction with my target design work, I have developed an automated shock
tuner for ICF implosion simulation.  A ICF target design requires launching a
sequence of shocks such that they merge within a limited time/space window.
Tuning the shock launch times is tedious and time consuming.  To assist my
target design work, developed optimization techniques that enable the
unsupervised tuning of these parameters.  Tuning work that used to occupy days
of attention can be completely automated and completed in hours.

Over the past 6 months, I have worked with the LLNL B-Division Low Temperature
Material Strength group on designing and fielding a broad range of experiments on the National
Ignition Facility (NIF), and Omega (Rochester, NY).
These include:
\begin{itemize}
	\item development of a bright 22 keV photon source (High Energy BackLighter
		``HEBL'') to be used for radiography in NIF material strength experiments
		of tantalum
	\item experiments on Omega using Laue diffraction to measure of the
		crystalline structure of thin (5 micron) tantalum foils subjected
		to 300 kBar shocks
\end{itemize}


\subsection*{Ph.D. Research}
My Ph.D. research studied the effect of different charged particle stopping
power models on the performance of inertial confinement fusion (ICF) targets.
Much work went into the development of a library (``Deeks'') that implements
many different models and does extensive consistency checking between the model
and the conditions to which it is applied.  This library has been integrated
into the pre-existing time dependent charged particle tracking (a Monte
Carlo-like transport scheme) package in an existing multi-physics code
(``Bucky'').  The integration allows me to examine the effect of different
transport models in integrated realistic simulations.

\vspace{0.1in}

More specifically, my research involves studying the effect collision operators have on transport properties of the plasma.  Due to the particularly high densities of ICF plasmas, this requires a careful, combined treatment of both the collisional and dielectric properties of the plasma, that is use of so-called ``convergent'' kinetic theories.  I have implemented stopping power models by 
Landau; 
Spitzer; 
Jackson; 
Brysk; 
Skupsky; 
Kihara and Aono; 
May and Cramer; 
Li and Petrasso; 
and 
Brown, Preston, and Singleton%
.
In my own research, I have derived and implemented Fermi-degenerate extensions of the Landau and Li and Petrasso models.  I am currently composing papers for publication on my Fermi-degenerate extension work and another on my analysis of the applicability of existing stopping power models in ICF ignition experiments.  


\subsection*{Software Development}
Over the course of my research, I have developed a significant amount of
scientific software.  I am a skilled Python developer, with 8 years of
experience with NumPy, SciPy, Matplotlib, Cython, PyTables, etc.  I am
proficient in Fortran, C, C++, and Go and have experience working in parallel
environments.  I am a vim, git, and \LaTeX user and am at home with the
command-line.

The computing aspect of my Ph.D. research required the integration of a library
written in C++ (Deeks) with an existing Fortran program (Bucky).  Data analysis
in compiled languages is overly time intensive, so I developed Python
interfaces to both codes.  The flexibility provided by the Python interface
enabled the rapid development of an automatic shock tuner for Bucky.  The tuner
completely automates a previously labor and analysis intensive process.  What
once took a week, can now be accomplished in hours.

\subsection*{Undergraduate Research at LLNL}
\begin{itemize}
	\item Summers 2003, 2004 - Worked with Ogden Jones on the analysis of ICF thinshell symmetry experiments using view-factor codes and radiation hydrodynamics code Hydra.
	\item Summer 2002 -  Worked with Dmitri Ryutov on magnetic islands in SSPX
\end{itemize}

\subsection*{Professional Societies and Affiliations}
\begin{itemize}
	\item The American Physical Society
	\item Software Carpentry
	\item The Hacker Within
	\item Scientific Computing in Python Conference
\end{itemize}

\pagebreak

\section*{Technical Leadership and Teaching}
\hrule
\vspace{0.05in}
\subsection*{Grant and Propsal Writing}
I am the LLNL Primary Investigator for a 3-year, \$6M proposal the US DoE
Office of Science to.  The proposal seeks funding to  experimentally investigate
the physics of shock reflection and the evolution of the Kelvin-Helmholtz
instability in conditions relevant to an ion driven fusion energy target
design.  If funded, I would be responsible for directing as \$1.1M in funds
over three years and managing a group of senior research scientists.  The
proposal is in collaboration with scientists at Lawrence Berkeley National
Laboratory, Lawrence Livermore National Laboratory, and the Princeton Plasma
Physics Laboratory.

\subsection*{Scientific Programming in Python Program Committee 2012-2013}
The Program Committee for the Scientific Programming in Python Conference is
responsible for reviewing and selecting talks from submitted proposals.
Committee is also responsible for reviewing submitted papers submitted to the
conference proceedings.

\subsection*{The Hacker Within}
The Hacker Within (THW) is a student-led, skill sharing interest group for
scientific software development.  It was founded in 2008 by two colleagues and
myself.  THW holds regular meetings organized around discussion of software
development.  I personally contributed talks on build systems (CMake), version
control (Subversion, Mercurial), text editors (VIM), and many Python-related
topics.

\vspace{0.1in}

THW also hosts occasional highly focused multi-day ``boot camp'' workshops. I
helped organize and teach boot camps on Unix literacy, C++, and Python in 2008,
2009, and 2010.  The Python boot camp, in particular, had more than 70 attendees
from 20 departments and included more than 18 hours of instruction.  In
addition to organization, I was responsible for teaching the sessions
``Modules, Scripts, Packages and Classes,''  ``Graphing with MatPlotLib,''
``Extending Python with C/C++ using SWIG,''  and ``Batteries Included - The
Python Standard Library.''

\subsection*{Software Carpentry}
Software Carpentry seeks to make scientists more effective by teaching them
basic computing skills.  They have adopted the THW-style boot camp teaching
methodology and have, over the past two years, taught over 50 workshops in the
US, Canada, and Europe.  I have assisted with several Software Carpentry
workshops at UC-Berkeley and Lawrence Berkeley Laboratory.  I am currently
participating in a Software Carpentry instructor training course covering the
academic literature on effective software development practice and techniques
for effective computing instruction.

\subsection*{Engineering Problem Solving}
At the University of Wisconsin-Madison, I was a Teaching Assistant for NE 271:
Engineering Problem Solving.  This laboratory course covers the solution of engineering
problems using commercially-available software tools (spreadsheets, symbolic
manipulators, and equation solvers). The emphasis of the course is on nuclear
engineering problems, including radioactive decay, nuclear cross sections,
scattering, and criticality.  In this freshmen-level class, I was responsible
for guiding the in-class exercises as well as real-time debugging of students'
programming errors and computing misconceptions.

\section*{Community Leadership}
\hrule
\vspace{0.05in}
\subsection*{The Journey Community Church Council}
From 2009-2010, I was been one of 5 members of The Journey Community's Church
Council.  Members of the council, are responsible for overseeing the church's
finances.  They are responsible for setting the  budget for the church and
making financial decisions for the church.  Additionally, they oversee the
finances of \href{http://www.beautifulchild.org}{Beautiful Child}, a charity
that sponsors an orphanage in Jacmel, Haiti.

\subsection*{Jumptown}
From 2002-2004 I was president of a swing dancing club.  I was responsible for
organizing weekly classes, monthly dances and semesterly workshops.  Under my
supervision the club grew from about 20 people to more than 100 and oversaw a
budget of several thousand dollars.

\section*{Awards}
\hrule
\vspace{0.05in}
\begin{itemize}
	\item National Physical Sciences Consortium Fellowship in conjunction with Sandia National Laboratories 2007-2010
	\item James Poukey Fellowship 2006
	\item National Academy for Nuclear Training Scholarship 2002-2004
	\item American Nuclear Society Scholarship 2002-2003
\end{itemize}

\bigskip

\end{document}
